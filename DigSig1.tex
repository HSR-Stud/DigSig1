\documentclass{scrartcl}

% Standard Header für Zusammenfassungen verwenden.
\input{header/zusammenfassung}
\input{header/hyperref}
\usepackage{tikz}
% Musste fuer idiotenseite hinzugefuegt werden
\usepackage{circuitikz}
\usepackage{multicol}
\usepackage{rotating}
\newcommand{\formelbuch}[1]{$_{\textcolor{red}{\mbox{\small{S#1}}}}$}
\newcommand{\arraystretchOriginal}{1}
\newcommand{\buchSeite}[1]{$_{\textcolor{red}{\mbox{\small{ S#1}}}}$}

\setDefaultArrayStretch{2}

% Titel und Autor
\title{DigSig1 Formelsammlung}
\subtitle{Dozent: G.Schuster, Buch: Introduction to Signal Processing, Orfanidis}
\author{Jürg Rast, Gian Claudio Köppel}


\begin{document}
\selectlanguage{english}
\begin{titlepage}
	\maketitle
	\thispagestyle{empty}
\end{titlepage}
\newpage

\tableofcontents
\newpage

\section{Random Signals\buch{Chapter 13}\buchSeite{713-719}}
\begin{tabularx}{\linewidth}{|l|X|}
	\hline
	probability distribution & $F_x(\alpha) = Pr\{x \leq \alpha \}$ \\
	\hline
	probability density & $f_x(\alpha) = \frac{d}{dx}F_x(\alpha)$\\
	\hline
	mean / expected value & $E\{x\} = \sum\limits_k \alpha_k Pr\{x = \alpha_k\} =
	\int\limits_{-\infty}^{\infty}a\cdot f_x(a)da$
	\\
	\hline
	signal power & $E\{ x^2 \} = \sum\limits_k \alpha_k^2 Pr\{x = \alpha_k\} =
	\int\limits_{-\infty}^{\infty} \alpha^2 f_x(\alpha) d\alpha$ \\
	\hline
	average absolute value & $E\{ |x| \} = \int\limits_{-\infty}^{\infty} |\alpha| f_x(\alpha) d\alpha 
										 = \int\limits_{0}^{\infty} \alpha[f_x(\alpha) + f_x(-\alpha)] d\alpha $
	\\ \hline
	variance $\sigma^2$ & 
	$\sigma_x^2 = E\{ [x-E\{ x \}]^2 \} = E(x^2) - E(x)^2 = \int\limits_{-\infty}^{\infty}[\alpha - E\{ x \}]^2 f_x(\alpha) d\alpha $
	\\ \hline
	autocorrelation function & 
	$R_{XX}(k) = E[x(n+k)x(n)]$
	\\ \hline
	power spectrum & 
	$S_{XX}(\omega) = \sum\limits_{k=-\infty}^{\infty} R_{XX}(k)e^{-j\omega k} \qquad \omega = \frac{2\pi f}{f_s} $
	\\ \hline
\end{tabularx}
\section{Sampling and Recostruction\buch{Chapter 1}}
\subsection{Analog Signals}
\begin{tabularx}{\linewidth}{|l|X|}
	\hline
	Fourier transform & $X(\Omega) = \int\limits_{-\infty}^{\infty} x(t)e^{-j\Omega t}dt \qquad \Omega = 2\pi f $ \\
	\hline
	inverse Fourier transform & $ x(t) = \int\limits_{-\infty}^{\infty} X(\Omega)e^{j\Omega t} \frac{d\Omega}{2 \pi} $ \\
	\hline
\end{tabularx}

\subsection{Digital Signals}
\subsubsection{Sampling Theorem}
Sampling means that the analog signal is periodically measured with a sampling interval T. The discrete index $n$, relates to
the time $t$ as follows:
\[ t = nT \qquad n = 0,1,2,\ldots \]
The sampling frequency relates to the sampling interval as follows:
\[ f_s = \frac{1}{T} \]
Sampling Theorem (Nyquist rate):
\[ f_s \geq 2f_{max} \qquad \text{or} \qquad T \leq \frac{1}{2f_{max}} \]
Nyquist interval:
\[ \left[-\frac{f_s}{2}, \frac{f_s}{2}\right] \]
\section{Quantization\buch{Chapter 2}}

\subsection{Quantization process}
\begin{tabular}{|l|l|l|}
	\hline
	$R$			& full-scale range		& $R = Q \cdot 2^B$
	\\ \hline
	$B$			& bits					& $B = log_2(\frac{R}{Q})$
	\\ \hline
	$Q$			& quantization width	& $Q = \frac{R}{2^B}$
	\\ \hline
	$e$			& quantization error	& $e_Q(nT) = x_Q(nT) -x(nT)$
	\\ \hline
	$e_{RMS}$	& root-mean-square error & $e_{RMS} = \frac{Q}{\sqrt{12}}$
	\\ \hline
	$SNR$		& signal-to-noise ratio	& $SNR = 20 log_{10}(\frac{R}{Q}) = 6B\, dB$
	\\ \hline
	$\sigma_e^2$& average power / variance & $\sigma_e^2 = E[e^2(n)] = \frac{Q^2}{12}$
	\\ \hline
\end{tabular}


\subsection{Oversampling and noise shaping}
\begin{tabular}{|l|l|l|}
	\hline
	$L$	& oversampling ratio	& $L = \frac{f_s'}{f_s}$ with $f_s'$ as higher sampling rate
	\\ \hline
	$\Delta B$	& saved bits without noise shaping	& $\Delta B = 0.5 \cdot log_2(L)$ \\
				& saved bits with noise shaping		& $\Delta B = (p + 0.5) \cdot log_2(L) - 0.5 \cdot log_2(\frac{\pi^{2p}}{2p + 1})$
	\\ \hline
\end{tabular}

Performance of oversampling noise shaping quantizers:

\setArrayStretch{1.2}
\begin{tabular}{|c|l|c|c|c|c|c|c|}
	\hline
	$p$	& $L$							& 4		& 8		& 16	& 32	& 64	& 128
	\\ \hline
	0	& $\Delta B =0.5 \cdot log_2(L)$		& 1.0	& 1.5	& 2.0	& 2.5	& 3.0	& 3.5 \\
	1	& $\Delta B =1.5 \cdot log_2(L) - 0.86$ & 2.1	& 3.6	& 5.1	& 6.6	& 8.1	& 9.6 \\
	2	& $\Delta B =2.5 \cdot log_2(L) - 2.14$	& 2.9	& 5.4	& 7.9	& 10.4	& 12.9	& 15.4 \\
	3	& $\Delta B =3.5 \cdot log_2(L) - 3.55$	& 3.5	& 7.0	& 10.5	& 14.0	& 17.5	& 21.0 \\
	4	& $\Delta B =4.5 \cdot log_2(L) - 5.02$	& 4.0	& 8.5	& 13.0	& 17.5	& 22.0	& 26.5 \\
	5	& $\Delta B =5.5 \cdot log_2(L) - 6.53$	& 4.5	& 10.0	& 15.5	& 21.0	& 26.5	& 32.0 \\
	\hline
\end{tabular}
\resetArrayStretch


\subsection{D/A converters}
\begin{tabular}{|l|l|}
	\hline
	natural binary & $x_Q = R(b_1 2^{-1} + b_2 2^{-2} + \ldots + b_B 2^{-B})$
	\\ \hline
	offset binary	& $x_Q = R(b_1 2^{-1} + b_2 2^{-2} + \ldots + b_B 2^{-B} - 0.5)$
	\\ \hline
	two's complement & $x_Q = R(\overline{b_1} 2^{-1} + b_2 2^{-2} + \ldots + b_B 2^{-B} - 0.5)$
	\\ \hline
\end{tabular}


\subsection{A/D converters}
\subsection{Analog and digital dither}
Dither is a small white noise signal that is added to the input signal befor quantization.
\section{Discret-Time Systems\buch{Chapter 3}}
\subsection{Linearity and time invariance}
\begin{tabular}{ll}
\textbf{Linearity:} & If $y(n) = x(n)$ then $y(2n)=2x(n)$\\
\textbf{Time invariance:} & If $y(n) = x(n)$ then $y(n+\delta) = x(n+\delta)$\\
\end{tabular}

\subsection{Impulse response}
\begin{tabular}{ll}
	\textbf{LTI form:}		& $y(n) = \sum\limits_m x(m)h(n-m)$ \\
	\textbf{direct form:}	& $y(n) = \sum\limits_m h(m)x(n-m)$
\end{tabular}


\subsection{FIR and IIR filters}
\begin{tabular}{|l|l|l|}
	\hline
	$M$	& filter order				& 
	\\ \hline
	$h$	& filter impulse response	& $\{ h_0, h_1, h_2, \ldots , h_M, 0, 0, \ldots\}$
	\\ \hline
	$L_h$	& length of $h$			& $L_h = M + 1$
	\\ \hline
	FIR		& FIR filtering equation	& $y(n) = \sum\limits_{m=0}^{M} h(m)x(n-m)$
	\\ \hline
	IIR		& IIR filtering equation	& $y(n) = \sum\limits_{m=0}^{\infty} h(m)x(n-m)$
	\\ \hline
\end{tabular}


\subsection{Causality}
\begin{tabular}{ll}
	\textbf{causal} & right sided signals, they are non-zero for $n>=0$ \\
	\textbf{anticausal} & left sided signals, they are non-zero for $n<=-1$ \\
	\textbf{mixed signals} & double-sided signals \\
\end{tabular}

\subsection{Stability}
\textbf{Stability Condition} $\sum\limits_{n=-\infty}^{\infty}\left| h(n)
\right| < \infty$\\
An LTI system is stable, if a bounded input can only generate bounded outputs.\\

\section{FIR Filtering and Convolution\buch{Chapter 4}}
\subsection{Block Processing Methods}
\subsubsection{Convolution}


\subsubsection{Direct Form}
\begin{tabular}{|l|l|}
	\hline
	$h$		& $h=[h_0,h_1, \ldots , h_M]$
	\\ \hline
	$L_h$	& $L_h = M + 1$
	\\ \hline
	$L_y$	& $L_y = L + M = L_x + L_h - 1$
	\\ \hline
	$y(n)$	& $y(n) = \sum\limits_{m=max(0,n-L+1)}^{min(n,M)} h(m)x(n-m)$
	\\ \hline 
\end{tabular}


\subsubsection{Convolution Table}


\subsubsection{LTI Form}
\[
	y(n) = \sum\limits_{m=max(0,n-M)}^{min(n,L-1)} x(m)h(n-m)
\]


\subsubsection{Matrix Form}
The convolutional eqations can also be written in the linear matrix form:
\[
	y = H  x	\qquad \text{or} \qquad y = Xh
\]
where $H$ is built out of the filter's impulse response $h$ or the signal matrix $X$ is built out of the input signal. 
The filter matrix $H$, respectivly the signal matrix $X$, must be rectangular with dimensions
\[
	L_y \times L_x = (L + M)\times L \qquad \text{or} \qquad
	L_y \times L_h = (L + M)\times (M + 1)
\]

Example:
\setArrayStretch{1}
\[
	y =
	\begin{bmatrix}
		y_0 \\
		y_1 \\
		y_3 \\
		y_4 \\
		y_5 \\
		y_6 \\
		y_7
	\end{bmatrix}
	= \begin{bmatrix}
		h_0	& 0		& 0		& 0		& 0 \\
		h_1	& h_0	& 0 	& 0 	& 0 \\
		h_2	& h_1	& h_0	& 0		& 0 \\
		h_3 & h_2	& h_1	& h_0	& 0 \\
		0	& h_3	& h_2	& h_1	& h_0 \\
		0	& 0		& h_3	& h_2	& h_1 \\
		0	& 0		& 0		& h_3	& h_2 \\
		0	& 0		& 0		& 0		& h_3	 
	  \end{bmatrix}
	  \begin{bmatrix}
	  	x_0 \\
	  	x_1 \\
	  	x_2 \\
	  	x_3 \\
	  	x_4
	  \end{bmatrix}
	= Hx
\]
\resetArrayStretch


\subsubsection{Flip-and-Slide Form}


\subsubsection{Transient and Steady-State Behavior}
\begin{tabular}{|l|l|}
	\hline
	input-on trainsients	& $ 0 \leq n < M $
	\\ \hline
	steady state			& $ M \leq n \leq L-1 $
	\\ \hline
	input-off transient		& $ L-1 < n \leq L-1+M $
	\\ \hline
\end{tabular}\newline

Therefore, the direct form takes the following different forms depending
on the value of the output index $n$:
\[
	y_n =
		\left\{
			\begin{array}{r l l l}
				\sum\limits_{m=0}^{n} & h_m x_{n-m}		& \quad \text{if } 0 \leq n < M			& \quad \text{input-on} \\
				\sum\limits_{m=0}^{M} & h_m x_{n-m}		& \quad \text{if } M \leq n \leq L-1	& \quad \text{steady state} \\
				\sum\limits_{m=n-L+1}^{M} & h_m x_{n-m}	& \quad \text{if } L-1 < n \leq L-1+M	& \quad \text{input-off}
			\end{array}
		\right.
\]


\subsubsection{Convolution of Infinite Sequences}
Three cases:
\setArrayStretch{1}
\begin{enumerate}
  \item Infinite filter, finite input; i.e., $M = \infty$, $L < \infty$
  \item Finite filter, infinite input; i.e., $M < \infty$, $L = \infty$
  \item Infinite filter, infinite input; i.e., $M = \infty$, $L = \infty$
\end{enumerate}
\resetArrayStretch

Therefore, the direct form takes the following different forms:
\[
	y_n =
		\left\{
			\begin{array}{r l l}
				\sum\limits_{m=max(0,n-L+1)}^{n} & h_m x_{n-m}		& \quad \text{if $M = \infty$, $L < \infty$} \\	
				\sum\limits_{m=0}^{min(n,M)} & h_m x_{n-m}		& \quad \text{if $M < \infty$, $L = \infty$ } \\
				\sum\limits_{m=0}^{n} & h_m x_{n-m}	& \quad \text{if $M = \infty$, $L = \infty$ }
			\end{array}
		\right.
\]




	
	
\section{z-Transformation\buch{Chapter 5}}

\subsection{Basic Properties}

\[
	X(z) = \sum\limits_{n=-\infty}^\infty x(n)z^{-n}
\]

\begin{tabularx}{0.6\textwidth}{|l|X|}
	\hline
	linearity & $a_1x_1(n) + a_2x_2(n) \overset{Z}{\longrightarrow} a_1X_1(z) + a_2X_2(z)$
	\\ \hline
	delay	& $x(n) \overset{Z}{\longrightarrow} X(z) \Longrightarrow x(n-D) \overset{Z}{\longrightarrow} z^{-D}X(z)$
	\\ \hline
	convolution & $y(n) = h(n) \convolution x(n) \Longrightarrow Y(z) = H(z)X(z)$
	\\ \hline
	modulation & $a^n g(n) \Longrightarrow G(\frac{z}{a})$
	\\ \hline
	time inversion & $g(-n) \Longrightarrow G(z^{-1})$
	\\ \hline
\end{tabularx}


\subsection{Region of Convergence}

\[
	\{ z \in \mathbb{C} \quad | \quad X(z) = \sum\limits_{n=-\infty}^{\infty} x(n)z^{-n} \neq \pm \infty \}
\]

\begin{tabularx}{0.6\textwidth}{|l|X|}
	\hline
	infinite geometric series 1 & $1 + x + x^2 + x^3 + \ldots = \sum\limits_{n=0}^{\infty} x^n = \frac{1}{1-x}$
	\\ \hline
	infinite geometric series 2 & $x + x^2 + x^3 + \ldots = \sum\limits_{m=1}^{\infty} x^m = \frac{x}{1-x}$
	\\ \hline
\end{tabularx}


\subsection{Causality and Stability}

For a signal or system to be simultaneously stable and causal, it is necessary that all its poles lie strictly inside the unit circle in the z-plane. 
\[ 1 > \max\limits_{i}|p_i| \]

A signal or system
can also be simultaneously stable and anticausal, but in this case all its poles must lie
strictly outside the unit circle.
\[ 1 < \min\limits_{i}|p_i| \]

\subsection{Frequency Spectrum}
\begin{align*}
	X(\omega)&= \sum\limits_{n=-\infty}^{\infty} x(n)e^{-j\omega n} \quad \text{(DTFT)} \
	\qquad \omega = \frac{2 \pi f}{f_s} \qquad -\pi \leq \omega \leq \pi
\end{align*}

\begin{align*}
	x(n)&= \frac{1}{2\pi} \int\limits_{-\pi}^{\pi} X(\omega)e^{j\omega n} d\omega \quad \text{(inverse DTFT)} 
\end{align*}

Another useful relationship is Parseval’s equation, which relates the total energy of
a sequence to its spectrum:

\begin{align*}
	\sum_{n=-\infty}^{\infty} |x(n)|^2&= \frac{1}{2\pi} \int\limits_{-\pi}^{\pi} |X(\omega)|^2 d\omega \quad \text{(Parseval)} 
\end{align*}

Some Z-Transforms: \\ \\
\begin{tabularx}{0.6\textwidth}{|X|X|X|}
	\hline
	\textbf{x(n} ) & \textbf{X(z}) & \textbf{ROC} \\
	\hline
	$u(n)$ & $\frac{1}{1 - z^{-1}}$ & $|z|>1$ \\
	\hline
	$-u(-n-1)$ & $\frac{1}{1 - z^{-1}}$ & $|z|<1$ \\
	\hline
	$-(-1)^n u(n)$ & $\frac{1}{1 + z^{-1}}$ & $|z|>1$ \\
	\hline
	$-(-1)^n u(-n-1)$ & $\frac{1}{1 + z^{-1}}$ & $|z|<1$ \\
	\hline
	$a^n u(n)$ & $\frac{1}{1 + az^{-1}}$ & $|z|>a$ \\
	\hline
	$-a^n u(-n-1)$ & $\frac{1}{1 + az^{-1}}$ & $|z|<a$ \\
	\hline
\end{tabularx}\\ \\

Euler: 

\begin{align*}
	\cos(\alpha) = \frac{e^{j\alpha} + e^{-j\alpha}}{2} \\
	\sin(\alpha) = \frac{e^{j\alpha} - e^{-j\alpha}}{2j} \\
	e^{j\frac{\pi}{2}n} = j^n\\
	e^{-j\frac{\pi}{2}n} = (-j)^n\\
\end{align*}


\section{Transfer Functions\buch{Chapter 6}}
\subsection{IIR-Form:\buchSeite{223,224}}
\begin{align}
H(z) = \frac{N(z)}{D(z)}
		= \frac{b_0 + b_1 z^{-1} + b_2 z^{-2} + \ldots + b_N z^{-N}}{1 + a_1 z^{-1} + a_2 z^{-2} + \ldots + a_M z^{-M} } && a_0 = 1 \text{ normalize to } 1 \notag
\end{align}

if $D(z) = 1$, the IIR Form can be reduced to a FIR Filter:
\[
	H(z) = N(z) = b_0 + b_1 z^{-1} + b_2 z^{-2} + \ldots + b_N z^{-N}
\]

Example: find $H(z)$ for $h(n) = \left[1,3,4,5\right]$
\[ H(z) = 1 + 3z^{-1} + 4z^{-2} + 5z^{-3}\]

Example: $y(n) = 0.25 \cdot y(n-2) + x(n) $
\[
	Y(z) = 0.25z^{-2}Y(z) + X(z)
\]

\subsection{Steady state response\buchSeite{230-232}}
\[\cos(\omega_0 n) \rightarrow \left|H(\omega_0) \right| \cos(\omega_0 n + \arg(H(\omega_0))) \]

\[\sin(\omega_0 n) \rightarrow \left|H(\omega_0) \right| \sin(\omega_0 n + \arg(H(\omega_0))) \]

\begin{tabularx}{0.6\textwidth}{|l|X|}
	\hline
	phase delay & $d(\omega) = - \frac{\arg(H(\omega))}{\omega}$\ \qquad $\arg H(\omega = -\omega d(\omega))$
	\\ \hline 
	group delay & $d_g(\omega) = -\frac{d}{d\omega}(\arg(H(\omega)))$	
	\\ \hline
\end{tabularx}\\ \\

\subsection{Transient Response\buchSeite{232}}
\begin{align}
&\text{complex sinuousness:} & & x(n) = e^{j \omega_0 n} \cdot u(n) && \Rightarrow && X(Z) = \frac{1}{1-e^{j \omega_0} z^{-1}} && ROC |z| > |e^{j\omega_0}|=1\notag\\ 
&\text{effective time:} && n_{eff} = \frac{\ln \epsilon}{\ln \rho} &&\text{with: }& &\rho = \max\limits_{i}\left|p_i \right| &&\notag\\
&\text{f and magnitude reponses:} && H(\omega)=\frac{b}{1-ae^{-j\omega}} &&\Rightarrow &&
\left| H(\omega)\right| = \frac{b}{\sqrt{1-2a\cos(\omega) + a^2}} &&\notag\\
&\text{requirement that:} && |H(\omega_0)|=1 &&&& &&\notag
\end{align}\\

\subsection{Unit Step Response\buchSeite{239}}
\begin{tabularx}{1\textwidth}{l X}
	DC-Gain: & $H(0) = H(z)|_{z=1}= \sum\limits_{n=0}^{\infty}h(n) $
	\\
	AC-Gain: & $H(\pi) = H(z)|_{z=-1}= \sum\limits_{n=0}^{\infty}(-1)^n h(n) $
\end{tabularx}\\ \\


\subsection{Pole/Zero Design\buchSeite{242,245}}
\begin{tabularx}{1\textwidth}{Xll}
	3dB width & $\Delta\omega \simeq 2(1-R)$ & $=:R$ is the pole \\
	full width at half maximum of the magnitude squared response & $|H(\omega)|^2=\frac{1}{2}|H(\omega_0)|^2=\frac{1}{2}$&
\end{tabularx}\\ \\

\subsubsection{2 pole conjugate filter\buchSeite{244-246}}

\begin{tabularx}{1\textwidth}{l X}
	poles: & $ p = R e^{j\omega}$ \qquad and \qquad $ p^* = R e^{-j\omega}$
	\\ 
	Transfer function: & $H(z)= \frac{G}{(1-R e^{-j\omega}z^{-1})(1-Re^{j\omega}z^{-1})}$\
	$=\frac{G}{1+a_1z^{-1}+a_2z^{-2}}$
	\\
	Parameter: & $a_1 = -2R\cos(\omega_0)$ \qquad; \qquad $a_2 = R^2$
	\\ 
	filter impulse Response & $h(n) = \frac{G}{\sin(\omega_0)}R^n \sin(\omega_0 n + \omega_0)$
	\\ 
	& $G = (1-R)\sqrt{1-2R\cos(2\omega_0)+ R^2}$
\end{tabularx}\\ \\

\subsubsection{2 pole 2 zero filter\buchSeite{228,249}}

\begin{tabularx}{1\textwidth}{l X}
	poles: & $ p = R e^{j\omega}$ \qquad and \qquad $ p^* = R e^{-j\omega}$
	\\ 
	zeros: & $ z_1 = r e^{j\omega}$ \qquad and \qquad $ z_1^* = r e^{-j\omega}$
	\\
	Transfer function: & $H(z)= \frac{(1-r e^{j\omega}z^{-1})(1-re^{-j\omega}z^{-1})}{(1-R e^{j\omega}z^{-1})(1-Re^{-j\omega}z^{-1})}$\
	$=\frac{1+b_1z^{-1}+b_2z^{-2}}{1+a_1z^{-1}+a_2z^{-2}}$
	\\ 
	Parameter: & $a_1 = -2R\cos(\omega_0)$ \qquad; \qquad $a_2 = R^2$
	\\ 
	& $b_1 = -2r\cos(\omega_0)$ \qquad; \qquad $b_2 = r^2$
\end{tabularx}\\ \\

\subsubsection{Notch and Comb Filter\buchSeite{249-251}}
The zeros of the filters located on the unit circle and the poles are in the unit
circle.

\begin{tabularx}{1\textwidth}{l X}
	Transfer function: & $H(z) = \frac{N(z)}{D(z)} $
	\\
	& $N(z) = \prod\limits_{i=1}^{M}(1- e^{j\omega_i}z^{-1)} $ \qquad (notch polynomial)
	\\
	& $D(z) = N(\rho^{-1})  = \prod\limits_{i=1}^{M}(1- e^{j\omega_i}\rho z^{-1)} $
	\\
	& $H(z) = \frac{N(z)}{(N\rho^{-1}z)}= \frac{1+b_1 z^{-1} + b_2z^{-2} + \ldots + b_M z^{-M}}
	{1+\rho b_1 z^{-1} + \rho^2 b_2z^{-2} + \ldots + \rho^M b_M z^{-M}} $\\
	&$a_i=\rho^ib_i \qquad$ mit $\qquad i=1,2,\ldots,M$ \\
	\end{tabularx}\\ \\



\section{Idiotenseite}
\input{idiotenseite/trigo/subsections/Winkelargumente}
\input{idiotenseite/trigo/subsections/Periodizitaet}
\input{idiotenseite/trigo/subsections/Quadrantenbeziehungen}
\input{idiotenseite/trigo/subsections/Additionstheoreme}
\input{idiotenseite/trigo/subsections/DoppelHalbwinkel}
\input{idiotenseite/trigo/subsections/Produkte}
\input{idiotenseite/trigo/subsections/SummeDifferenzen}
\input{idiotenseite/trigo/subsections/Euler}
\input{idiotenseite/diverses/subsections/Reihenentwicklung}

\end{document}
 
